\documentclass[a4paper,oneside,12pt]{report}


\usepackage[utf8]{inputenc}
\usepackage[T1]{fontenc}
\usepackage[ngerman]{babel}
\usepackage{graphicx}
\usepackage{setspace}

\graphicspath{ {./Images/} }


%Adjust the page margins 
\usepackage[left=3cm, right=3cm, top=2cm, bottom=2cm, a4paper, portrait]{geometry}

%For fancy headers and footers
\usepackage{fancyhdr}
\usepackage{mathpazo}
%Change captions
\usepackage[font=footnotesize,labelfont=bf,tableposition=top]{caption}
%Enable link support in the pdf
\usepackage{hyperref}
%Using BibLaTeX (replacing BibTeX)
\usepackage[autostyle,german=guillemets]{csquotes}
\usepackage[natbib=false,citestyle=numeric-comp,bibstyle=numeric-comp,sortcites=true,backend=biber]{biblatex}
\bibliography{template_advanced_thesis_references}


%Adjust headings and footers --> \usepackage{fancyhdr}
%Give the headings some space
\setlength{\headheight}{15pt}
%This is valid for all pages exept chapters
\pagestyle{fancy}
\fancyhf{} % clear all headers and footers
\rhead[\thepage]{{\leftmark}}
\rfoot[{\leftmark}]{\thepage}
\renewcommand{\headrulewidth}{0.4pt}
% For \chapters \maketitle
\fancypagestyle{plain}{%
	\fancyhf{} % clear all header and footer fields
	\rfoot[{\leftmark}]{\thepage}%
	\renewcommand{\headrulewidth}{0pt}
	\renewcommand{\footrulewidth}{0pt}
}




%Adjust hyperlink behavior -> \usepackage{hyperref}
\hypersetup{
	%bookmarks=true,        										 % show bookmarks bar?
	unicode=false,         											 % non-Latin characters in Acrobatics bookmarks
	pdftoolbar=true,      											 % show Acrobatics toolbar?
	pdfmenubar=true,       											 % show Acrobatics menu?
	pdffitwindow=false,     										 % window fit to page when opened
	pdfstartview={FitH},   											 % fits the width of the page to the window
	pdftitle={FWP Grundlagen der kuenstlichen Intelligenz},    		 % title
	pdfauthor={Philipp Muhr, Michael Mican, Maximilian Anzinger}, 	 % author
	pdfsubject={Ampelphasenerkennung},    							 % subject of the document
	pdfkeywords={Kuenstliche Intelligenz, Informatik}, 				 % list of keywords
	pdfnewwindow=true,     											 % links in new window
	colorlinks=true,        										 % false: boxed links; true: colored links
	linkcolor=black,       											 % color of internal links
	citecolor=black,      											 % color of links to bibliography
	filecolor=black,      											 % color of file links
	urlcolor=black        											 % color of external links
}


%Absatzzeilen verhindern
\clubpenalty = 10000
\widowpenalty = 10000 
\displaywidowpenalty = 10000




\begin{document}

	\pagenumbering{alph}  % numbering a, b, c
	\pagenumbering{arabic} % numbering arabic
	
\begin{titlepage}
	\begin{center}
		
		\textsc{\scshape \huge Prüfungsstudienarbeit (PStA)}\\
		\rule{1\textwidth}{1mm} \\[0.5cm]
		{ \LARGE  {\bfseries Technische Hochschule Deggendorf}}\\[0.5cm]
		{ \LARGE   Fakultät Elektrotechnik, Medientechnik und Informatik}\\[0.5cm]
		\rule{1\textwidth}{1mm} \\[4cm] 
		{ \Large \bfseries Thema: Ampelphasenerkennung}\\[2cm]
		{ Fach: FWP Grundlagen der künstlichen Intelligenz}\\[5cm]
		
		\begin{minipage}{0.4\textwidth}
			\begin{flushleft} \normalsize
				\emph{vorgelegt von:}\\[0.3cm]
				Philipp Muhr, Michael Mican, Maximilian Anzinger\\[0.2cm]
				Deggendorf,  \today
			\end{flushleft}
		\end{minipage}
		\begin{minipage}{0.5\textwidth}
			\begin{flushright} \normalsize
				\emph{Prüfer:}\\[0.3cm]
				Prof. Dr. Andreas \textsc{Fischer}\\[0.2cm]
			\end{flushright}
		\end{minipage}\\[2cm]
		{\large \today}
				
	\end{center}
\end{titlepage}

	
	%Inhaltsverzeichnis
	\tableofcontents
	\cleardoublepage
	
	
	\chapter{Einleitung}
	%Link zu den Kapiteln inkl. Inhaltsverzeichnis
	\begin{onehalfspace}
		Im Straßenverkehr finden sich viele Herausforderungen. Es müssen viele Situationen vorausgesehen werden, um eine sichere Fahrt zu gewährleisten. Bei immer höher werdenden Verkehrsaufkommen werden es die Fahrer auch in Zukunft nicht einfach haben. Damit die Verkehrssituation sicherer gestaltet werden kann, werden Unmengen an verschiedenen Fahrassistenzsystemen entwickelt. Diese Studienarbeit behandelt ein Assistenzsystem zur leichteren Ampelphasenwahrnehmung. Eine rote Ampel wird schnell übersehen, daher muss auch in dieser Thematik unterstützte Wahrnehmung begünstigt werden. Es soll ein neuronales Netz trainiert werden, das Ampeln und deren unterschiedlichen Schaltphasen (rot, rot-gelb, gelb und grün) erkennen kann. Ziel dieser künstlichen Intelligenz soll sein, dass Sie den Fahrer durch Statusmeldungen in beispielsweise dem Bordmonitor mit der aktuellen Ampelphase informiert. Falls die Sicht des Mobilisten eingeschränkt ist und dieser die Phasen der Ampel nur schwer bis gar nicht erkennen kann soll er sich auf die Anzeige der Ampelerkennung im Bordmonitor verlassen können. Diese Arbeit wird Aufschluss geben, wie die künstliche Intelligenz aufgebaut und trainiert wurde. Außerdem werden unterschiedliche Verfahren im Bereich Image processing beschrieben, um eventuell performantere Ergebnisse zu erzielen. Im ersten Kapitel wird dargelegt, auf welchem neuronalen Netz aufgebaut wurde und wie dieses angelegt wurde.
	\end{onehalfspace}

	\chapter{Netzwerk}
	\begin{onehalfspace}
		Wie bereits erwähnt soll die Erkennung der Ampel bzw. weiterführend die Phase dieser durch ein neuronales Netz realisiert werden.\\
		Mittlerweile gibt es viele verschiedene Neuronale Netze zur Erkennung von Objekten in Bildern und/oder Videos (Object Recognition). Als Basis dient hier meist ein sog. Convolutional Neural Network (CNN) welches aus mehreren namensgebenden Convolutional Layern, Pooling Layern, einem Fully Connected Layer und einer Softmax Funktion besteht.
		Dieses Konzept wird dann um verschiedene Operatoren, Funktionen oder Ähnliches erweitert.\\
		Im Nachfolgenden wird das, von uns genutzte, Netzwerk YoloV3-tiny genauer erklärt bzw. allgemein auf das Konzept von YOLO eingegangen.
		\section{YOLO}
		\subsection{Yolo allgemein}
		YOLO ist laut offizieller Aussage ein "`Echtzeit Objekterkennungs System"
		\footnote{"https://pjreddie.com/darknet/yolo/", besucht am 30.05.2020 (eigene Übersetzung)},
		welches vorallem auf Geschwindigkeit bei trotzdem gut bleibender Genauigkeit setzt.\\
		YOLO nutzt im Gegensatz zu vielen herkömmlichen Objekterkennungsnetzen (wie z.B. Faster-RCNN oder Mask-RCNN) nicht mehrerer Netze hintereinander, bei dem z.B. das erste durch ein sog. Region Proposal Network dargestellt wird (welches interessante Regionen herausfiltert), sondern geht "`einen völlig anderen Weg und schaut nur ein einziges Mal auf das Bild"
		\footnote{"https://www.sigs-datacom.de/trendletter/2018-10/4-wie-man-in-echtzeit-mehrere-objekte-mit-deep-learning-und-yolo-erkennen-und-klassifizieren-kann.html", besucht am 30.05.2020}
		und "`erkennt gleichzeitig alle interessanten Bereiche und klassifiziert diese mit eine[m] Score"
		\footnote{"https://www.sigs-datacom.de/trendletter/2018-10/4-wie-man-in-echtzeit-mehrere-objekte-mit-deep-learning-und-yolo-erkennen-und-klassifizieren-kann.html", besucht am 30.05.2020}\\
		Diese Technik ermöglicht es dem Netzwerk eine deutlich schneller Klassifizierung vorzunehmen. \clearpage
		Jedoch bringt diese Geschwindigkeit bzw. diese Technik auch einen Nachteil mit sich. Für ein solches Model ist es schwieriger ein sehr kleines Objekt zu erkennen, da das Grid welches auf das Bild gelegt wird sehr "grob"\space im Vergleich zu anderen Netzen ist. \\
		Deshalb ist ein Model wie z.B. Retinanet mit einem Backbone wie einem FPN deutlich besser in der Erkennung von kleinen Objekten bzw. auch in der Genauigkeit der Vorhersagen generell, auch wenn dies natürlich Performance kostet. (vgl. Abbildung~\ref{fig:comparison}
		\footnote{"https://syncedreview.com/2018/03/27/the-yolov3-object-detection-network-is-fast/", besucht am 30.05.2020})\\
		\begin{figure}[h!]
			\includegraphics[width=\linewidth]{Comparison.png}
			\caption{Yolov3 Comparison}
			\label{fig:comparison}
		\end{figure}
		\subsection{YoloV3-tiny}
		Für unseren Einsatzfall lohnt es sich, wie oben beschrieben, einen Blick auf Tiny Yolo zu werfen. \\
		Ein "`ganzes"\space YOLO Netzwerk wie z.B. das neue YoloV4 oder das gängige YoloV3 Model bieten gute Ergebnisse, jedoch ist die Performance dieser auf einem Nvidia Jetson Nano Gerät leider zu schlecht, um diese dem Nutzer direkt "`live"\space weitergeben zu können.
		Daher wird eine "`kleine"\space Version von Yolo (genannt Tiny-Yolo) für diesen Anwendungsfall genutzt. Tiny-Yolo (V3) kann Bilder auf einem Nvidia Jetson Nano mit ca. 25 Bildern pro Sekunde klassifzieren und bietet somit eine, für diese Platform, sehr gute Performance.\clearpage
		Aufgrund dieser deutlich höheren Performance (ca. 6,2x die Geschwindigkeit von YoloV3-416) verschlechtert sich die Genauigkeit jedoch auch um ca. 40\% (vgl. Abbildung~\ref{fig:yolov3-tiny})\\
		\begin{figure}[h!]
			\includegraphics[scale=0.7]{YoloV3-tiny.png}
			\caption{Yolov3(-tiny) Comparison}
			\label{fig:yolov3-tiny}
		\end{figure}
	\end{onehalfspace}
	\chapter{Realisierung}
	\begin{onehalfspace}
		Wie schon in der vorherigen Sektion erwähnt, soll das Projekt auf einem Nvidia Jetson Nano realisiert werden. Dieser bietet für seinen vergleichsweise niedrigen Preis (ca. 99\$
		\footnote{"https://nvidianews.nvidia.com/news/nvidia-announces-jetson-nano-99-tiny-yet-mighty-nvidia-cuda-x-ai-computer-that-runs-all-ai-models", besucht am 31.05.2020})
		im Hinblick auf Inferenz eine sehr gute Performance für die Kosten.
		\section{Hardware}
		Der Nvidia Jetson Nano berechnet mit seiner 128 Kern Nvidia Maxwell Grafikeinheit und seinem Quad-Core ARM Cortex -A57 laut Nvidia 472 GFLOPs.
		\footnote{"https://developer.nvidia.com/embedded/develop/hardware", besucht am 31.0.5.2020}
		Das Gerät wird für unseren Anwendungsfall zusätzlich mit einer Raspberry Pi Camera v2.1 ausgestattet, welche den Videostream liefert. Das Nvidia System ermöglicht eine Video Dekodierung von bis zu vier gleichzeitigen Full-HD Streams bei 30 Bildern pro Sekunde bzw. analog von acht bei 30 Bildern pro Sekunde.
		\footnote{"https://developer.nvidia.com/embedded/develop/hardware", besucht am 31.0.5.2020}
		\section{Software}
		Nvidia schafft es bei der Jetson Platform die Geräte sehr gut zu optimieren bzw. die Hardware optimal zu nutzen. Dies ermöglichen Frameworks wie z.B. Deepstream und/oder TensorRT
		\subsection{Deepstream}
		Deepstream ist ein Software Development Kit, das "`ein beschleunigtes AI Framework darstellt, welches zur Erstellung von Intelligenten Video Analyse Pipelines genutzt wird."
		\footnote{"https://docs.nvidia.com/metropolis/deepstream/dev-guide/", besucht am 31.05.2020 (eigene Übersetzung)}
		Deepstream fungiert als "`Schnittstelle" zwischen Python, C, C++ Applikationen und der Platform selbst (z.B. einem Nvidia Jetson Nano / TX2).
		Weiterführend basiert es auf CUDA (einer GPU basierten parallelen Programmiersprache welche oft bei neuronalen Netzen zum Einsatz kommt) welches TensorRT nutzt.
		\subsection{TensorRT}
		TensorRT ist, analog zu Deepstream, auch ein Software Development Kit, welches für "high-performance deep learning inference"
		\footnotetext{"https://developer.nvidia.com/tensorrt", besucht am 31.05.2020} genutzt werden kann.
		Applikationen die auf TensorRT basieren können bis zu 40 mal schneller sein als vergleichbare Applikation, welche auf einem CPU berechnet werden.
		\footnotetext{"https://developer.nvidia.com/tensorrt", besucht am 31.05.2020}
		Dieses SDK konvertiert ein trainiertes Neuronales Netzwerk mithilfe eines TensorRT Optimizers in eine TensorRT Runtime Engine. Dies geschieht durch die Kombinierung von Layern und optimierung der Kernel Auswahl.
		\footnotetext{"https://docs.nvidia.com/deeplearning/tensorrt/developer-guide/index.html"}
		Durch diese Umwandlung bzw. Optimierung des Netzwerkes kann ein enormer Geschwindigkeitsgewinn , wie oben beschrieben, erzielt werden.
	\end{onehalfspace}
	\chapter{Labeling}
	\begin{onehalfspace}
		Labeling
	\end{onehalfspace}
	\chapter{Data Preprocessing}
	\begin{onehalfspace}
		Data Preprocessing
	\end{onehalfspace}
	\chapter{Image Processing}
	\begin{onehalfspace}
		Das Image Processing beschreibt den Vorgang der Aufbereitung des Bildes mithilfe der Daten der KI. Die ursprüngliche Idee war durch diesen Verarbeitungsschritt die Phase der erkannten Ampel zu identifizieren. \newline 
		Beim ersten Versuch wurden die Detektierten Bounding boxen (Bbox) aus dem Bild herausgeschnitten um so nur noch ein Bild der Ampel zu haben. Der Bildausschnitt wird dann entsättigt und so zu einem Schwarz-Weiß Bild konvertiert (vgl. Abbildung~\ref{fig:TLCrop}). Der Ausschnitt wird dann gleichmäßig in drei Sektoren unterteilt, welche jeweils ein drittel des Bildes abdecken. Es wird nun der Durchschnitts "weiß-" Wert jedes Sektors berechnet und Validiert. Bei diesem Verarbeitungsschritt wird überprüft ob die Analyse ein plausibles Ergebnis geliefert hat. Wenn diese Überprüfung des Datensatzes negativ ausfällt werden nun die original Farbkanäle (von vor der Entsättigung) benutzt. Dabei werden im ersten drittel nur die Rot werte, im zweiten nur die Rot und Grün Kanäle im selben Verhältnis und im dritten Sektor nur die grün werte. Diese werte werden dann auf den selben Wertebereich normalisiert und dann analog zur Entsättigten Variante analysiert und validiert. Das Ergebniss der Analyse setzt sich dann aus den eindeutigsten werten beider Untersuchungsarten zusammen. \newline
		Diese Ergebnisse erreichten eine Trefferquote von etwa 40\%. Die meisten Fehldetektionen können dadurch begründet werden, dass die Bboxen nicht genau deckend die grenzen der Ampel kennzeichnen und somit die Hintergrundlandschaft hinter der Ampel auf die Analyse Einfluss nimmt. Um dem Entgegenzuwirken wird nur ein 10 Pixel breites band in der Mitte des Bildes analysiert. Somit werden die Ränder aus der Detektion entfernt. (vgl. Abbildung~\ref{fig:TLCrop}) Diese Verbesserung hebt die Detektionsquote auf etwa 60\%.\newline
		Die Zuschneidung in Y-Richtung gestaltet sich komplizierter. Hierfür wird eine Pixel Spalte im oberen viertel des Bildes in der Mitte betrachtet. Das dunkelste Pixel markiert dann den Obersten beziehungsweise Untersten punkt. Um keine "Pixelausreiser", welche durch das Hintergrundrauschen verursacht werden, als grenze zu detektieren, werden bei Detektion die folgenden 10 Pixel mit überprüft. Es is möglich, dass der dunkelste Bereich im Bild nicht die grenze der Ampel markiert. Um dem Entgegen zu wirken wird der Median aus allen dunklen Pixeln gezogen und ab dem ersten Pixel mit dieser Helligkeit abgeschnitten. (vgl. Abbildung~\ref{fig:TLCrop}) Damit konnte die Detektionsquote leicht verbessert werden.\newline
		\begin{figure}[h!]
			\includegraphics[width=\linewidth]{TLCropping.jpg}
			\caption{Analyse schritte}
			\label{fig:TLCrop}
		\end{figure}
		\section{Probleme}
		Dieser Ansatz bringt unzufrieden stellende Ergebnisse. Viele Detektionen sind Falsch oder knapp richtig (unterschiede der Durchschnitte der Sektionen gering). Die Zuschneidung in x-Richtung muss sich auf eine möglichst gleichmäßige Detektion verlassen, da der Ausschnitt ansonsten nicht mittig der Ampel befindet. Die Zuschneidung in y-Richtung ist abhängig der licht und Umgebungsverhältnisse mal genauer mal weniger genau und so nicht robust einsetzbar. Weitere Probleme bei der Detektion der grenzen entstehen, wenn die Ampel nicht Schwarz ist, was in einigen Ländern, wie zum Beispiel dem Vereinigten Königreiches, häufig vorkommt. Hinzu kommt, dass dieses System nur funktional ist, wenn die Ampel aufrecht hängt. Aus diesen Gründen wurde dieser Versuch für das finale Ergebnis nicht mit einbezogen und es wurde eine Lösung mit einem neuronalen Netz umgesetzt.
		
		
	\end{onehalfspace}
	\chapter{Data augmentation}
	\begin{onehalfspace}
		Data Augmentation
	\end{onehalfspace}
	\chapter{Probleme}
	\begin{onehalfspace}
		Probleme
	\end{onehalfspace}
	\chapter{Schlusswort}
	\begin{onehalfspace}
		Schlusswort
	\end{onehalfspace}
	
\end{document}
