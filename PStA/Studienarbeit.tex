\documentclass[a4paper,oneside,12pt]{report}


\usepackage[utf8]{inputenc}
\usepackage[T1]{fontenc}
\usepackage[ngerman]{babel}
\usepackage{graphicx}
\usepackage{setspace}


%Adjust the page margins 
\usepackage[left=3cm, right=3cm, top=2cm, bottom=2cm, a4paper, portrait]{geometry}

%For fancy headers and footers
\usepackage{fancyhdr}
\usepackage{mathpazo}
%Change captions
\usepackage[font=footnotesize,labelfont=bf,tableposition=top]{caption}
%Enable link support in the pdf
\usepackage{hyperref}
%Using BibLaTeX (replacing BibTeX)
\usepackage[autostyle,german=guillemets]{csquotes}
\usepackage[natbib=false,citestyle=numeric-comp,bibstyle=numeric-comp,sortcites=true,backend=biber]{biblatex}
\bibliography{template_advanced_thesis_references}


%Adjust headings and footers --> \usepackage{fancyhdr}
%Give the headings some space
\setlength{\headheight}{15pt}
%This is valid for all pages exept chapters
\pagestyle{fancy}
\fancyhf{} % clear all headers and footers
\rhead[\thepage]{{\leftmark}}
\rfoot[{\leftmark}]{\thepage}
\renewcommand{\headrulewidth}{0.4pt}
% For \chapters \maketitle
\fancypagestyle{plain}{%
	\fancyhf{} % clear all header and footer fields
	\rfoot[{\leftmark}]{\thepage}%
	\renewcommand{\headrulewidth}{0pt}
	\renewcommand{\footrulewidth}{0pt}
}




%Adjust hyperlink behavior -> \usepackage{hyperref}
\hypersetup{
	%bookmarks=true,        										 % show bookmarks bar?
	unicode=false,         											 % non-Latin characters in Acrobatics bookmarks
	pdftoolbar=true,      											 % show Acrobatics toolbar?
	pdfmenubar=true,       											 % show Acrobatics menu?
	pdffitwindow=false,     										 % window fit to page when opened
	pdfstartview={FitH},   											 % fits the width of the page to the window
	pdftitle={FWP Grundlagen der kuenstlichen Intelligenz},    		 % title
	pdfauthor={Philipp Muhr, Michael Mican, Maximilian Anzinger}, 	 % author
	pdfsubject={Ampelphasenerkennung},    							 % subject of the document
	pdfkeywords={Kuenstliche Intelligenz, Informatik}, 				 % list of keywords
	pdfnewwindow=true,     											 % links in new window
	colorlinks=true,        										 % false: boxed links; true: colored links
	linkcolor=black,       											 % color of internal links
	citecolor=black,      											 % color of links to bibliography
	filecolor=black,      											 % color of file links
	urlcolor=black        											 % color of external links
}


%Absatzzeilen verhindern
\clubpenalty = 10000
\widowpenalty = 10000 
\displaywidowpenalty = 10000




\begin{document}

	\pagenumbering{alph}  % numbering a, b, c
	\pagenumbering{arabic} % numbering arabic
	
\begin{titlepage}
	\begin{center}
		
		\textsc{\scshape \huge Prüfungsstudienarbeit (PStA)}\\
		\rule{1\textwidth}{1mm} \\[0.5cm]
		{ \LARGE  {\bfseries Technische Hochschule Deggendorf}}\\[0.5cm]
		{ \LARGE   Fakultät Elektrotechnik, Medientechnik und Informatik}\\[0.5cm]
		\rule{1\textwidth}{1mm} \\[4cm] 
		{ \Large \bfseries Thema: Ampelphasenerkennung}\\[2cm]
		{ Fach: FWP Grundlagen der künstlichen Intelligenz}\\[5cm]
		
		\begin{minipage}{0.4\textwidth}
			\begin{flushleft} \normalsize
				\emph{vorgelegt von:}\\[0.3cm]
				Philipp Muhr, Michael Mican, Maximilian Anzinger\\[0.2cm]
				Deggendorf,  \today
			\end{flushleft}
		\end{minipage}
		\begin{minipage}{0.5\textwidth}
			\begin{flushright} \normalsize
				\emph{Prüfer:}\\[0.3cm]
				Prof. Dr. Andreas \textsc{Fischer}\\[0.2cm]
			\end{flushright}
		\end{minipage}\\[2cm]
		{\large \today}
				
	\end{center}
\end{titlepage}

	
	%Inhaltsverzeichnis
	\tableofcontents
	\cleardoublepage
	
	
	\chapter*{Einleitung}
	%Link zu den Kapiteln inkl. Inhaltsverzeichnis
	\addcontentsline{toc}{chapter}{Einleitung}
	\begin{onehalfspace}
		Im Straßenverkehr finden sich viele Herausforderungen. Es müssen viele Situationen vorausgesehen werden, um eine sichere Fahrt zu gewährleisten. Bei immer höher werdenden Verkehrsaufkommen werden es die Fahrer auch in Zukunft nicht einfach haben. Damit die Verkehrssituation sicherer gestaltet werden kann, werden Unmengen an verschiedenen Fahrassistenzsystemen entwickelt. Diese Studienarbeit behandelt ein Assistenzsystem zur leichteren Ampelphasenwahrnehmung. Eine rote Ampel wird schnell übersehen, daher muss auch in dieser Thematik unterstützte Wahrnehmung begünstigt werden. Es soll ein neuronales Netz trainiert werden, das Ampeln und deren unterschiedlichen Schaltphasen (rot, rot-gelb, gelb und grün) erkennen kann. Ziel dieser künstlichen Intelligenz soll sein, dass Sie den Fahrer durch Statusmeldungen in beispielsweise dem Bordmonitor mit der aktuellen Ampelphase informiert. Falls die Sicht des Mobilisten eingeschränkt ist und dieser die Phasen der Ampel nur schwer bis gar nicht erkennen kann soll er sich auf die Anzeige der Ampelerkennung im Bordmonitor verlassen können. Diese Arbeit wird Aufschluss geben, wie die künstliche Intelligenz aufgebaut und trainiert wurde. Außerdem werden unterschiedliche Verfahren im Bereich Image processing beschrieben, um eventuell performantere Ergebnisse zu erzielen. Im ersten Kapitel wird dargelegt, auf welchem neuronalen Netz aufgebaut wurde und wie dieses angelegt wurde.
	\end{onehalfspace}

	\chapter*{YOLO}
	\addcontentsline{toc}{chapter}{YOLO}
	\begin{onehalfspace}
		test
	\end{onehalfspace}
	
\end{document}
